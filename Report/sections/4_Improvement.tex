\clearpage
\section{Thảo luận và Đề xuất (Discussion and Recommendations)}
\label{Sec:improvement}

Sau khi hoàn thành việc xây dựng và đánh giá các mô hình dự đoán ý định mua hàng, phần này diễn dịch sâu hơn
ý nghĩa của kết quả, chuyển hóa chúng thành các hàm ý kinh doanh cụ thể cho ShopSmart E-commerce, đồng thời
thảo luận hạn chế và các hướng cải tiến để mô hình có thể triển khai thực tế một cách đáng tin cậy.

%=================================================================================
\subsection{Diễn dịch kết quả (Result Interpretation)}
\label{subsec:result_interpretation}

\subsubsection{Phân tích hiệu suất mô hình}

Kết quả so sánh mô hình (Bảng~\ref{tab:model_metrics}) cho thấy nhóm \textbf{ensemble dựa trên decision tree}
(Gradient Boosting, XGBoost, Random Forest) đạt hiệu quả tổng thể tốt nhất. Nhận xét này phù hợp với trực quan hóa
tại Hình~\ref{fig:fig1_f1_auc}, khi các mô hình này cùng lúc có \textbf{F1} và \textbf{ROC-AUC} cao hơn rõ rệt so với
các mô hình tuyến tính hoặc dựa trên khoảng cách (KNN). Điều này hàm ý dữ liệu hành vi có nhiều \textit{quan hệ phi tuyến}
và \textit{tương tác thuộc tính}.

\begin{table}[H]
\centering
\caption{Kết quả đánh giá các mô hình trên tập kiểm thử (test set).}
\label{tab:model_metrics}
\begin{tabular}{lccccc}
\toprule
\textbf{Mô hình} & \textbf{Accuracy} & \textbf{Precision} & \textbf{Recall} & \textbf{F1-Score} & \textbf{ROC-AUC} \\
\midrule
Gradient Boosting      & 0.901865 & 0.725806 & 0.589005 & 0.650289 & 0.927936 \\
XGBoost                & 0.892133 & 0.682390 & 0.568063 & 0.620000 & 0.919550 \\
Random Forest          & 0.897405 & 0.729537 & 0.536649 & 0.618401 & 0.917758 \\
Decision Tree          & 0.863747 & 0.561828 & 0.547120 & 0.554377 & 0.734453 \\
Support Vector Machine & 0.884428 & 0.701245 & 0.442408 & 0.542536 & 0.873646 \\
Logistic Regression    & 0.881184 & 0.743169 & 0.356021 & 0.481416 & 0.887273 \\
K-Nearest Neighbors    & 0.868613 & 0.636792 & 0.353403 & 0.454545 & 0.772459 \\
Naive Bayes            & 0.676399 & 0.297665 & 0.801047 & 0.434043 & 0.797426 \\
\bottomrule
\end{tabular}
\end{table}

\begin{figure}[H]
  \centering
  \includegraphics[width=0.9\textwidth]{images/visualizations_20251230_135903/fig1_f1_auc.png}
  \caption{So sánh F1-Score và ROC-AUC giữa các mô hình.}
  \label{fig:fig1_f1_auc}
\end{figure}

\textbf{1. Gradient Boosting là mô hình tốt nhất theo tiêu chí F1-Score và ROC-AUC.}
Trong thí nghiệm hiện tại, Gradient Boosting đạt \textbf{F1-Score cao nhất} và \textbf{ROC-AUC cao nhất} (Bảng~\ref{tab:model_metrics}),
cho thấy mô hình không chỉ phân loại tốt tại ngưỡng mặc định mà còn có khả năng \textit{ranking} xác suất Purchase tốt.
Về mặt triển khai, đây là lợi thế quan trọng vì doanh nghiệp có thể chọn \textbf{threshold} linh hoạt theo mục tiêu
(ví dụ: ưu tiên bắt nhiều Purchase hay ưu tiên giảm báo động giả).

\textbf{2. Random Forest là lựa chọn cạnh tranh khi ưu tiên tính ổn định và diễn giải.}
Random Forest có hiệu năng xấp xỉ nhóm dẫn đầu (đặc biệt về Accuracy/Precision và ROC-AUC), đồng thời thuận lợi khi cần
giải thích theo hướng \textit{feature importance} và kiểm soát rủi ro mô hình. Trong bối cảnh doanh nghiệp yêu cầu
\textit{interpretability} cao (ví dụ: cần giải thích cho đội marketing hoặc quản trị rủi ro), Random Forest là một lựa chọn thực dụng.

\textbf{3. Trade-off giữa Precision và Recall thể hiện rõ qua không gian Precision--Recall.}
Hình~\ref{fig:fig2_precision_recall} cho thấy các mô hình có “thiên hướng” khác nhau:
\begin{itemize}
  \item \textbf{Naive Bayes} có Recall rất cao nhưng Precision thấp, phù hợp khi hành động can thiệp rẻ và mục tiêu là \textit{tối đa hóa phát hiện Purchase} (chấp nhận nhiều FP).
  \item \textbf{Logistic Regression} có Precision cao nhưng Recall thấp, phù hợp khi FP có chi phí lớn (ví dụ tặng voucher/ưu đãi tốn kém), nhưng có nguy cơ bỏ sót nhiều entries Purchase.
  \item \textbf{Gradient Boosting} đạt cân bằng tốt nhất giữa Precision và Recall trong nhóm hiệu năng cao, nên là lựa chọn hợp lý nếu mục tiêu tối ưu \textbf{F1}.
\end{itemize}

\begin{figure}[H]
  \centering
  \includegraphics[width=0.9\textwidth]{images/visualizations_20251230_135903/fig2_precision_recall.png}
  \caption{Không gian Precision--Recall của các mô hình.}
  \label{fig:fig2_precision_recall}
\end{figure}

\textbf{4. Diễn giải lỗi dự đoán thông qua Confusion Matrix.}
Với mô hình tốt nhất, Hình~\ref{fig:fig3_confusion_matrix_gb} cho thấy:
\begin{itemize}
  \item \textbf{FP} tương đối thấp (tỷ lệ báo động giả nhỏ), giúp hạn chế chi phí nếu dự đoán Purchase kéo theo ưu đãi.
  \item \textbf{FN} vẫn đáng kể, nghĩa là vẫn còn một phần entries mua thật bị bỏ sót.
\end{itemize}

\begin{figure}[H]
  \centering
  \includegraphics[width=0.9\textwidth]{images/visualizations_20251230_135903/fig3_confusion_matrix_gb.png}
  \caption{Confusion Matrix của mô hình Gradient Boosting.}
  \label{fig:fig3_confusion_matrix_gb}
\end{figure}

Kết hợp với ROC-AUC cao, điều này gợi ý mô hình có chất lượng \textit{ranking} tốt nhưng \textbf{threshold mặc định}
chưa chắc tối ưu. Do đó, một khuyến nghị quan trọng là thực hiện \textbf{threshold tuning} theo chi phí FP/FN
hoặc theo mục tiêu kinh doanh (ví dụ: tối ưu lợi nhuận kỳ vọng thay vì tối ưu Accuracy/F1).

\subsubsection{Vai trò của class imbalance và hướng xử lý}

Dữ liệu có \textbf{class imbalance} (tỷ lệ Purchase thấp). Trong các bài toán như vậy, mô hình thường dễ nghiêng về lớp đa số,
dẫn đến \textbf{Recall(Purchase)} bị hạn chế nếu không có cơ chế hiệu chỉnh.
Trong notebook, phần kết luận kỹ thuật cũng khuyến nghị các kỹ thuật xử lý class imbalance cho môi trường production
(ví dụ: \textit{class weights}, SMOTE, hoặc các ensemble phù hợp).

\textbf{Khuyến nghị kỹ thuật (định hướng cải tiến):}
\begin{itemize}
  \item \textbf{Class weights / cost-sensitive learning:} ưu tiên giảm FN nếu mục tiêu là bắt nhiều Purchase.
  \item \textbf{Resampling (SMOTE / oversampling / undersampling có kiểm soát):} tăng khả năng học đặc trưng lớp Purchase,
        nhưng cần kiểm soát overfitting và leakage trong pipeline.
  \item \textbf{Đánh giá bổ sung bằng PR-AUC:} trong bối cảnh class imbalance, PR-AUC thường phản ánh chất lượng mô hình hữu ích hơn ROC-AUC.
\end{itemize}

\subsubsection{Diễn giải tầm quan trọng thuộc tính (Feature Importance) và insight hành vi}

Phân tích EDA và phần \textit{feature importance} (đối với các mô hình tree-based) cho thấy các tín hiệu mạnh nhất liên quan đến Purchase gồm:
\textbf{PageValues}, \textbf{ExitRates}, \textbf{BounceRates}, \textbf{ProductRelated\_Duration}, và \textbf{ProductRelated}.
Các biến phân loại quan trọng gồm \textbf{Month}, \textbf{VisitorType}, và \textbf{Weekend}.
Các insight này có giá trị vì vừa \textit{predictive} vừa \textit{actionable}:
\begin{itemize}
  \item \textbf{PageValues} cao thường phản ánh hành trình gần chuyển đổi.
  \item \textbf{ExitRates/BounceRates} cao là tín hiệu trải nghiệm kém hoặc nội dung/định hướng chưa phù hợp.
  \item \textbf{ProductRelated\_Duration} cao phản ánh mức độ cân nhắc mua lớn hơn.
  \item \textbf{Returning Visitor} có xu hướng chuyển đổi cao hơn.
  \item \textbf{Month} cho thấy yếu tố mùa vụ/chiến dịch (ví dụ các tháng cao điểm như November/May theo EDA).
\end{itemize}

\subsubsection{Tính tái lập kết quả (Reproducibility)}

Bên cạnh notebook, dự án có \texttt{validate\_analysis.py} để chạy kiểm chứng pipeline theo hướng “hồi quy kết quả”:
tải dữ liệu, tiền xử lý, huấn luyện nhiều mô hình, tổng hợp bảng metric và phân tích thuộc tính.
Đây là điểm cộng quan trọng trong bối cảnh học thuật bậc cao học vì giúp giảm phụ thuộc vào trạng thái notebook, đồng thời
tạo nền tảng cho kiểm thử tự động (CI) khi mở rộng dự án.

%=================================================================================
\subsection{Hàm ý kinh doanh (Business Implications)}
\label{subsec:business_implications}

Dựa trên các tín hiệu dự đoán quan trọng và hành vi quan sát được, phần này đề xuất các chiến lược có thể triển khai cho ShopSmart E-commerce.
Lưu ý: các KPI nêu dưới đây là mục tiêu định hướng và cần được hiệu chỉnh qua A/B testing.

\subsubsection{1. Tối ưu hóa Page Value}

\textbf{Cơ sở}: \textbf{PageValues} là tín hiệu mạnh nhất liên quan Purchase, phản ánh mức “tiến gần chuyển đổi”.

\textbf{Đề xuất}:
\begin{itemize}
  \item Phân rã nhóm trang có PageValues cao để xác định yếu tố thành công (nội dung, bố cục, CTA, tốc độ tải).
  \item Tối ưu các trang có PageValues thấp: cải thiện CTA, thông tin sản phẩm, trust signals (đổi trả, bảo hành, đánh giá).
  \item Rà soát và tối ưu funnel checkout nhằm giảm \textit{friction}.
\end{itemize}

\textbf{KPI mục tiêu (gợi ý)}: tăng trung bình PageValues 15--20\% trong 6 tháng, đo lường kèm conversion rate và revenue/session.

\subsubsection{2. Giảm Exit Rate và Bounce Rate}

\textbf{Cơ sở}: \textbf{ExitRates} và \textbf{BounceRates} cao là tín hiệu tiêu cực và liên quan mạnh đến giảm xác suất Purchase.

\textbf{Đề xuất}:
\begin{itemize}
  \item Tối ưu tốc độ tải trang và hiệu năng mobile (giảm bounce do chậm).
  \item Cải thiện điều hướng và tìm kiếm: filter/sort, suggestion, autocomplete.
  \item Thử nghiệm \textit{exit-intent} (popup/offer) cho nhóm entries có xác suất rời đi cao.
  \item A/B testing landing page theo từng nguồn traffic.
\end{itemize}

\textbf{KPI mục tiêu (gợi ý)}: giảm ExitRates và BounceRates theo percentile (p50/p75) thay vì chỉ trung bình để tránh bị outlier chi phối.

\subsubsection{3. Tăng cường tương tác với trang sản phẩm}

\textbf{Cơ sở}: \textbf{ProductRelated} và \textbf{ProductRelated\_Duration} tăng tương quan với Purchase.

\textbf{Đề xuất}:
\begin{itemize}
  \item Nâng chất lượng trang sản phẩm: ảnh, video, mô tả, so sánh, FAQ, review.
  \item Gợi ý sản phẩm theo hành vi (recommendations) và theo ngữ cảnh (similar/compatible items).
  \item Thiết kế hành trình khám phá: collections theo chủ đề, “Xem thêm”, cross-sell/upsell hợp lý.
\end{itemize}

\textbf{KPI mục tiêu (gợi ý)}: tăng số trang sản phẩm xem/entry 20--30\% và tăng thời gian ở trang sản phẩm nhưng vẫn kiểm soát drop-off.

\subsubsection{4. Nhắm mục tiêu Returning Visitors}

\textbf{Cơ sở}: \textbf{Returning Visitor} có tỷ lệ chuyển đổi cao hơn.

\textbf{Đề xuất}:
\begin{itemize}
  \item Loyalty program và ưu đãi theo mức độ trung thành.
  \item Email/SMS/push cá nhân hóa theo hành vi (browse abandonment, cart abandonment).
  \item Remarketing theo phân khúc: high-intent vs low-intent.
\end{itemize}

\textbf{KPI mục tiêu (gợi ý)}: tăng tỷ lệ returning users và tăng conversion rate của returning segment.

\subsubsection{5. Tối ưu hóa theo mùa vụ (Seasonality)}

\textbf{Cơ sở}: biến \textbf{Month} phản ánh seasonality; EDA cho thấy một số tháng có tỷ lệ Purchase cao hơn.

\textbf{Đề xuất}:
\begin{itemize}
  \item Lập kế hoạch tồn kho, nội dung, ngân sách ads theo tháng cao điểm.
  \item Tạo chiến dịch kích cầu tháng thấp điểm (bundle, free-ship, limited-time offer).
  \item Chuẩn hóa dashboard theo tháng để phát hiện drift theo mùa.
\end{itemize}

\textbf{KPI mục tiêu (gợi ý)}: tăng uplift doanh thu mùa cao điểm, đồng thời giảm biến động tháng thấp điểm.

%=================================================================================
\subsection{Đề xuất triển khai hệ thống dự đoán (Deployment Recommendations)}
\label{subsec:deployment}

Ba hạng mục triển khai tối thiểu (mang tính thực dụng) gồm:

\subsubsection{Thu thập dữ liệu (near real-time)}
\begin{itemize}
  \item Tracking events (GA/SDK nội bộ) và tổng hợp theo \textbf{entry}.
  \item Lưu trữ vào data store (stream + warehouse) để phục vụ inference và phân tích.
\end{itemize}

\subsubsection{Xử lý và dự đoán (preprocess + inference)}
\begin{itemize}
  \item Pipeline tiền xử lý đúng chuẩn huấn luyện (one-hot, scaling) và kiểm soát schema.
  \item Model serving qua REST/gRPC; trả về \textbf{p(Purchase)} thay vì nhãn cứng.
  \item Cân nhắc \textbf{calibration} (Platt/isotonic) trước khi ra quyết định theo threshold.
\end{itemize}

\subsubsection{Hành động (decisioning) và đo lường}
\begin{itemize}
  \item Rule engine dựa trên xác suất:
    \begin{itemize}
      \item $p > 0.70$: ưu tiên upsell/premium recommendation.
      \item $0.30 \le p \le 0.70$: hiển thị ưu đãi nhẹ hoặc gợi ý sản phẩm phù hợp.
      \item $p < 0.30$ và có dấu hiệu rời đi: exit-intent coupon ở mức chi phí kiểm soát.
    \end{itemize}
  \item A/B testing bắt buộc để đo uplift (conversion, AOV, revenue, churn).
\end{itemize}

%=================================================================================
\subsection{Hạn chế của nghiên cứu (Limitations)}
\label{subsec:limitations}

\subsubsection{Hạn chế về dữ liệu}
\begin{itemize}
  \item \textbf{Nguồn dữ liệu đơn lẻ}: dataset thu thập từ một website TMĐT (bối cảnh Thổ Nhĩ Kỳ) nên tính khái quát hóa sang thị trường khác có thể hạn chế \cite{sakar2018}.
  \item \textbf{Tính thời điểm}: dữ liệu theo một giai đoạn lịch sử; hành vi người dùng có thể thay đổi theo thời gian (thiết bị, UX, kênh ads).
  \item \textbf{Theo dõi theo entry}: dữ liệu ở mức entry, chưa mô tả đầy đủ hành vi xuyên nhiều entries của cùng người dùng (thiếu user journey).
  \item \textbf{Thiếu ngữ cảnh sản phẩm}: không có category/price/promotion chi tiết nên khó phân tích theo danh mục và chiến lược giá.
  \item \textbf{Thiếu nhân khẩu học}: không có biến demographic (tuổi, giới tính, thu nhập) để phân khúc sâu.
\end{itemize}

\subsubsection{Hạn chế về phương pháp}
\begin{itemize}
  \item \textbf{Đánh giá hold-out}: thí nghiệm dựa trên một lần chia train/test; nên bổ sung Stratified K-Fold để kết luận vững hơn.
  \item \textbf{Hyperparameter tuning}: mô hình sử dụng cấu hình cơ bản; cần tuning (Randomized/Bayesian) cho top-model.
  \item \textbf{Chưa tối ưu theo kinh tế}: hiện tối ưu theo metric ML; triển khai thực tế nên tối ưu theo utility (profit uplift) và chi phí FP/FN.
  \item \textbf{Class imbalance chưa xử lý trong training}: cần thử class weights/resampling và đánh giá PR-AUC, calibration.
\end{itemize}

\subsubsection{Hạn chế về triển khai}
\begin{itemize}
  \item \textbf{Bối cảnh giả định}: ShopSmart là tình huống mô phỏng; cần kiểm chứng qua dữ liệu vận hành thực.
  \item \textbf{Chưa có A/B testing thực nghiệm}: mọi khuyến nghị kinh doanh cần A/B testing để đo uplift và kiểm soát rủi ro.
  \item \textbf{Chưa đánh giá scalability}: inference real-time ở lưu lượng lớn cần đánh giá latency, throughput, và chi phí hạ tầng.
\end{itemize}
