\clearpage
\section{Thảo luận và Đề xuất (Discussion and Recommendations)}
\label{Sec:improvement}

Sau khi hoàn thành việc xây dựng và đánh giá các mô hình dự đoán, phần này sẽ phân tích sâu hơn về ý nghĩa của kết quả, đưa ra các đề xuất kinh doanh cụ thể cho ShopSmart E-commerce, và thảo luận về những hạn chế của nghiên cứu.

%=================================================================================
\subsection{Diễn dịch kết quả (Result Interpretation)}
\label{subsec:result_interpretation}

\subsubsection{Phân tích hiệu suất mô hình}

Dựa trên kết quả thực nghiệm, chúng ta có thể rút ra những nhận định quan trọng sau:

\textbf{1. Random Forest là mô hình hiệu quả nhất}

Random Forest đạt hiệu suất cao nhất trong hầu hết các cấu hình thử nghiệm. Điều này có thể giải thích bởi:
\begin{itemize}
    \item Khả năng xử lý cả thuộc tính số và phân loại một cách hiệu quả.
    \item Cơ chế ensemble giúp giảm overfitting so với Decision Tree đơn lẻ.
    \item Khả năng nắm bắt các mối quan hệ phi tuyến tính trong dữ liệu clickstream.
    \item Robust với nhiễu và outliers trong dữ liệu.
\end{itemize}

\textbf{2. Ảnh hưởng của việc xử lý mất cân bằng lớp}

Kết quả cho thấy sự cải thiện đáng kể về Recall khi áp dụng các kỹ thuật xử lý mất cân bằng lớp:
\begin{itemize}
    \item \textbf{Dữ liệu gốc}: Mô hình có xu hướng dự đoán lớp đa số (không mua hàng), dẫn đến Recall thấp cho lớp mua hàng.
    \item \textbf{SMOTE}: Cải thiện Recall đáng kể bằng cách tạo ra các mẫu tổng hợp, giúp mô hình học được đặc điểm của khách hàng mua hàng tốt hơn.
    \item \textbf{Oversampling}: Cũng cải thiện Recall nhưng có nguy cơ overfitting cao hơn do sao chép các mẫu thiểu số.
\end{itemize}

\textbf{3. Trade-off giữa Precision và Recall}

Một quan sát quan trọng là sự đánh đổi (trade-off) giữa Precision và Recall:
\begin{itemize}
    \item Khi tăng Recall (phát hiện được nhiều khách hàng có ý định mua hơn), Precision có xu hướng giảm (tăng số lượng dự đoán sai).
    \item SMOTE thường cung cấp sự cân bằng tốt hơn giữa hai độ đo này so với Random Oversampling.
\end{itemize}

\subsubsection{Phân tích tầm quan trọng thuộc tính}

Dựa trên phân tích Feature Importance từ Random Forest, các thuộc tính quan trọng nhất được xác định như sau:

\textbf{Top 5 thuộc tính quan trọng:}
\begin{enumerate}
    \item \textbf{PageValues}: Thuộc tính quan trọng nhất, phản ánh giá trị trung bình của các trang được khách hàng truy cập. Giá trị cao cho thấy khách hàng đang xem các trang có liên quan đến quyết định mua hàng.

    \item \textbf{ExitRates}: Tỷ lệ thoát trang cao cho thấy khách hàng có xu hướng rời bỏ website, giảm khả năng mua hàng.

    \item \textbf{BounceRates}: Tương tự ExitRates, tỷ lệ thoát ngay (chỉ xem một trang rồi rời đi) là dấu hiệu tiêu cực.

    \item \textbf{ProductRelated\_Duration}: Thời gian xem trang sản phẩm càng lâu, khả năng mua hàng càng cao.

    \item \textbf{ProductRelated}: Số lượng trang sản phẩm được xem cũng là một chỉ báo tích cực.
\end{enumerate}

\textbf{Các thuộc tính phân loại quan trọng:}
\begin{itemize}
    \item \textbf{Month}: Tháng 11 (November) có tỷ lệ mua hàng cao nhất, liên quan đến Black Friday và các chương trình khuyến mãi cuối năm.
    \item \textbf{VisitorType}: Khách hàng quay lại (Returning Visitor) có tỷ lệ chuyển đổi cao hơn.
    \item \textbf{Weekend}: Phiên truy cập vào cuối tuần có hành vi khác biệt so với ngày thường.
\end{itemize}

%=================================================================================
\subsection{Hàm ý kinh doanh (Business Implications)}
\label{subsec:business_implications}

Dựa trên kết quả phân tích và mô hình hóa, chúng tôi đề xuất các chiến lược kinh doanh cụ thể cho ShopSmart E-commerce:

\subsubsection{1. Tối ưu hóa Page Value}

\textbf{Vấn đề}: PageValues là thuộc tính dự đoán quan trọng nhất, nhưng nhiều trang trên website hiện có giá trị thấp.

\textbf{Đề xuất:}
\begin{itemize}
    \item Phân tích các trang có PageValue cao để xác định các yếu tố thành công.
    \item Tối ưu hóa nội dung và thiết kế các trang có PageValue thấp.
    \item Thêm các yếu tố thúc đẩy hành động (call-to-action) vào các trang quan trọng.
    \item Cải thiện quy trình checkout để giảm friction.
\end{itemize}

\textbf{KPI mục tiêu}: Tăng trung bình PageValue lên 20\% trong 6 tháng.

\subsubsection{2. Giảm Exit Rate và Bounce Rate}

\textbf{Vấn đề}: Tỷ lệ thoát trang và tỷ lệ bỏ ngay cao là dấu hiệu của trải nghiệm người dùng kém.

\textbf{Đề xuất:}
\begin{itemize}
    \item Triển khai exit-intent popup với ưu đãi hấp dẫn cho khách hàng có xu hướng rời đi.
    \item Cải thiện tốc độ tải trang (page load time) để giảm bounce rate.
    \item Tối ưu hóa thanh điều hướng (navigation) để khách hàng dễ dàng tìm kiếm sản phẩm.
    \item Cung cấp nội dung liên quan (related content) để giữ chân khách hàng.
    \item A/B testing các landing page để xác định phiên bản hiệu quả nhất.
\end{itemize}

\textbf{KPI mục tiêu}: Giảm Exit Rate trung bình xuống 15\% và Bounce Rate xuống 10\%.

\subsubsection{3. Tăng cường tương tác với trang sản phẩm}

\textbf{Vấn đề}: ProductRelated và ProductRelated\_Duration là các chỉ báo quan trọng, nhưng nhiều khách hàng chỉ xem ít trang sản phẩm.

\textbf{Đề xuất:}
\begin{itemize}
    \item Cải thiện chất lượng hình ảnh và mô tả sản phẩm.
    \item Thêm video sản phẩm và đánh giá từ khách hàng.
    \item Triển khai hệ thống gợi ý sản phẩm (product recommendations) dựa trên hành vi duyệt web.
    \item Sử dụng infinite scroll hoặc ``Xem thêm sản phẩm'' để khuyến khích khách hàng khám phá.
    \item Tạo các bộ sưu tập sản phẩm (collections) theo chủ đề.
\end{itemize}

\textbf{KPI mục tiêu}: Tăng trung bình số trang sản phẩm xem mỗi phiên lên 30\%.

\subsubsection{4. Nhắm mục tiêu khách hàng quay lại}

\textbf{Vấn đề}: Returning Visitors có tỷ lệ chuyển đổi cao hơn nhưng chiếm tỷ lệ nhỏ.

\textbf{Đề xuất:}
\begin{itemize}
    \item Triển khai chương trình khách hàng thân thiết (loyalty program).
    \item Sử dụng email marketing cá nhân hóa để đưa khách hàng quay lại.
    \item Cung cấp ưu đãi đặc biệt cho khách hàng quay lại (returning customer discounts).
    \item Triển khai remarketing qua Facebook Ads và Google Ads.
    \item Gửi thông báo về sản phẩm trong ``wishlist'' hoặc giỏ hàng bị bỏ.
\end{itemize}

\textbf{KPI mục tiêu}: Tăng tỷ lệ khách hàng quay lại lên 25\%.

\subsubsection{5. Tối ưu hóa theo mùa vụ}

\textbf{Vấn đề}: Tỷ lệ mua hàng biến động theo tháng, với đỉnh điểm vào tháng 11 (November).

\textbf{Đề xuất:}
\begin{itemize}
    \item Lập kế hoạch marketing chi tiết cho các tháng cao điểm (tháng 5, 11, 12).
    \item Chuẩn bị hàng tồn kho và nhân lực cho Black Friday và mùa lễ hội.
    \item Triển khai các chương trình khuyến mãi đặc biệt cho các tháng thấp điểm để kích thích nhu cầu.
    \item Tạo các sự kiện mua sắm theo chủ đề cho từng mùa.
\end{itemize}

\textbf{KPI mục tiêu}: Tăng doanh thu tháng cao điểm lên 40\% so với trung bình năm.

%=================================================================================
\subsection{Đề xuất triển khai hệ thống dự đoán (Deployment Recommendations)}
\label{subsec:deployment}

Để áp dụng mô hình vào thực tế, ShopSmart E-commerce cần triển khai hệ thống dự đoán theo các bước sau:

\subsubsection{Kiến trúc hệ thống đề xuất}

\begin{enumerate}
    \item \textbf{Thu thập dữ liệu thời gian thực}:
    \begin{itemize}
        \item Tích hợp Google Analytics hoặc công cụ tracking tương tự.
        \item Thu thập dữ liệu clickstream trong thời gian thực.
        \item Lưu trữ dữ liệu phiên truy cập vào database.
    \end{itemize}

    \item \textbf{Xử lý và dự đoán}:
    \begin{itemize}
        \item Pipeline tiền xử lý dữ liệu tự động.
        \item Model serving với REST API.
        \item Dự đoán xác suất mua hàng cho mỗi phiên.
    \end{itemize}

    \item \textbf{Hành động tự động}:
    \begin{itemize}
        \item Nếu xác suất mua cao ($>$ 70\%): Đề xuất sản phẩm premium, upselling.
        \item Nếu xác suất mua trung bình (30-70\%): Hiển thị ưu đãi, giảm giá.
        \item Nếu xác suất mua thấp ($<$ 30\%) và có dấu hiệu rời đi: Popup exit-intent với coupon.
    \end{itemize}
\end{enumerate}

\subsubsection{Lộ trình triển khai}

\textbf{Giai đoạn 1: Pilot Testing}
\begin{itemize}
    \item Triển khai thử nghiệm với 10\% traffic.
    \item A/B testing so sánh với hệ thống hiện tại.
    \item Đánh giá tác động lên conversion rate và doanh thu.
\end{itemize}

\textbf{Giai đoạn 2: Mở rộng}
\begin{itemize}
    \item Triển khai cho toàn bộ website.
    \item Tích hợp với hệ thống CRM hiện có.
    \item Huấn luyện đội ngũ marketing sử dụng insights từ mô hình.
\end{itemize}

\textbf{Giai đoạn 3: Tối ưu hóa liên tục}
\begin{itemize}
    \item Thu thập feedback và cải thiện mô hình.
    \item Huấn luyện lại mô hình định kỳ (hàng tháng hoặc hàng quý).
    \item Mở rộng tính năng với personalization nâng cao.
\end{itemize}

%=================================================================================
\subsection{Hạn chế của nghiên cứu (Limitations)}
\label{subsec:limitations}

Nghiên cứu này có một số hạn chế cần được lưu ý:

\subsubsection{Hạn chế về dữ liệu}

\begin{itemize}
    \item \textbf{Nguồn dữ liệu đơn lẻ}: Dataset được thu thập từ một website thương mại điện tử tại Thổ Nhĩ Kỳ, có thể không hoàn toàn phản ánh hành vi khách hàng Việt Nam.

    \item \textbf{Thời gian thu thập}: Dữ liệu được thu thập trong khoảng thời gian 1 năm (2017-2018), có thể đã lỗi thời so với hành vi người dùng hiện tại.

    \item \textbf{Session-based tracking}: Dữ liệu chỉ theo dõi từng phiên truy cập riêng lẻ, không theo dõi hành vi xuyên suốt của cùng một người dùng qua nhiều phiên.

    \item \textbf{Thiếu thông tin sản phẩm}: Không có dữ liệu chi tiết về loại sản phẩm, giá cả, hoặc danh mục sản phẩm.

    \item \textbf{Thiếu thông tin nhân khẩu học}: Không có thông tin về độ tuổi, giới tính, thu nhập của khách hàng.
\end{itemize}

\subsubsection{Hạn chế về phương pháp}

\begin{itemize}
    \item \textbf{Thuật toán giới hạn}: Chỉ thử nghiệm 4 thuật toán cơ bản, chưa khám phá các phương pháp nâng cao như Deep Learning, Gradient Boosting (XGBoost, LightGBM).

    \item \textbf{Feature engineering cơ bản}: Chưa thực hiện các kỹ thuật feature engineering nâng cao như tạo các tương tác thuộc tính (feature interactions), polynomial features.

    \item \textbf{Hyperparameter tuning hạn chế}: Việc tối ưu hóa siêu tham số được thực hiện thủ công, chưa sử dụng các kỹ thuật tự động như Grid Search hoặc Bayesian Optimization.

    \item \textbf{Thiếu đánh giá kinh tế}: Chưa tính toán chi phí-lợi ích (cost-benefit analysis) cụ thể của việc triển khai mô hình.
\end{itemize}

\subsubsection{Hạn chế về triển khai}

\begin{itemize}
    \item \textbf{Môi trường giả định}: Bối cảnh ShopSmart E-commerce là hư cấu, chưa được kiểm chứng trong môi trường thực tế.

    \item \textbf{Thiếu A/B testing thực tế}: Các đề xuất kinh doanh cần được kiểm chứng qua A/B testing trước khi triển khai rộng rãi.

    \item \textbf{Vấn đề scalability}: Chưa đánh giá khả năng mở rộng của mô hình khi xử lý hàng triệu phiên truy cập.
\end{itemize}

